% Template for ICIP-2018 paper; to be used with:
%          spconf.sty  - ICASSP/ICIP LaTeX style file, and
%          IEEEbib.bst - IEEE bibliography style file.
% --------------------------------------------------------------------------
\documentclass{article}
\usepackage{res/packages/spconf,amsmath,graphicx}
\graphicspath{{res/imgs/},{res/plots/}}
\usepackage{hyperref}
\usepackage{url}
\newcommand{\link}[1]{\href{#1}{\textit{#1}}}
% Example definitions.
% --------------------
\def\x{{\mathbf x}}
\def\L{{\cal L}}

\renewcommand{\thesection}{\Roman{section}}
\renewcommand{\thesubsection}{\roman{subsection}}
\renewcommand{\thesubsubsection}{\alph{subsubsection}}





% Title.
% ------
\title{Website analyse}
%
% Single address.
% ---------------
\name{Romain Graux (28681700)
\thanks{Thanks to my mum.}}
\address{}
%
% For example:
% ------------
%\address{School\\
%	Department\\
%	Address}
%
% Two addresses (uncomment and modify for two-address case).
% ----------------------------------------------------------
%\twoauthors
%  {A. Author-one, B. Author-two\sthanks{Thanks to XYZ agency for funding.}}
%	{School A-B\\
%	Department A-B\\
%	Address A-B}
%  {C. Author-three, D. Author-four\sthanks{The fourth author performed the work
%	while at ...}}
%	{School C-D\\
%	Department C-D\\
%	Address C-D}
%
\begin{document}
%\ninept
%
\maketitle
%
\begin{abstract}
This report contains an analyzis of the website \link{miniclip.com}. The analyze covers three particular aspects: \textit{HTTP}, \textit{DNS} and \textit{TCP} packets.
\end{abstract}
%
% \begin{keywords}
% One, two, three, four, five
% \end{keywords}
%
\section{Introduction}
\label{sec:intro}
\subsection{Website}
\label{sub:web}
\textit{Minclip} is a free online games website. It was launched in 2001 by \textit{Robert Small} and \textit{Tihan Presbie}. The first thing to mention is that \link{miniclip.com} is automatically redirected to \link{www.miniclip.com/games/en/}, it will be explained in \textbf{\nameref{sec:HTTP}} but the analyze will be about this redirected url.
\subsection{Tools}
\label{sub:tools}
To perform the analyze, I need somme tools :
\begin{itemize}
    \item[--] \textit{HTTP}: I used \textit{Firefox Web Developper} tools.
    \item[--] \textit{DNS}: I used \textit{dig} command and \link{www.atlas.ripe.net}
    \item[--] \textit{TCP}: I used \textit{WireShark}
\end{itemize}

\section{HTTP}
\label{sec:HTTP}

\subsection{Headers}
\label{sub:headers}

As said in \nameref{sub:web}, the url \link{http://miniclip.com} is automatically redirected to \link{https://www.miniclip.com/games/en/}, this behavior is due to my browser and will not act necessarily the same on a different browser. 

First, the client will send an \textit{HTTP/2.0} request to the server that contains inter alia \texttt{Upgrade-Insecure- Requests} field set to \texttt{1}. As mention on \textit{MDN} website \cite{upgrade-insecure-request}, this field tells to the server that, if it is available, it should redirect the client to the \textit{HTTPS} version and \texttt{Status Code} returns \texttt{301}, this field means that it has to be redirected to the URL stored in the \texttt{location} field \cite{status-code-301}, so let's move to \link{https://www.miniclip.com} .

When we are on \link{https://www.miniclip.com}, the client will send an \textit{HTTP/2.0} request to the server that once again contains the \texttt{Upgrade-Insecure-Requests} field set to \texttt{1} but also contains the \texttt{Accept-Language} field set to \texttt{en} (in my case). It will act the same as aforementioned, the server will set the \texttt{Status Code} field to \texttt{301} and set the \texttt{location} field to \link{https://www.miniclip.com/games/en/} and we are redirected.

Then the client send the \textit{request header} to the server that will return a \texttt{200 Status Code} and the
source code of the page stored in the body, the client is now able to load the content.

Every HTTPS request is established over the port 443 and every HTTP request is established over the port 80.

\subsection{Resources types}
\label{sub:res}

When the client as load the HTML, he needs to load all the content linking in the file from different domain (\url{www.youtube.com}, \url{apis.google.com}, \url{static.miniclipcdn.com}, , ...). For that, it sends requests to all server that contains data, for the main page, it sends 94 requests. As we are on \textit{HTTP/2.0}, \textit{multiplexing} is used and one connection is established between the client and a domain containing data the client needs to download.
% The website need a lot of images to illustrate games you can choose.

\begin{figure}[h]
    \centering
    \includegraphics[width=0.45\textwidth]{res/imgs/resources.png}
    \caption{Resources types}
    \label{fig:res}
\end{figure}

\section{DNS}
\label{sec:DNS}


\section{TCP}
\label{sec:TCP}





% References should be produced using the bibtex program from suitable
% BiBTeX files (here: strings, refs, manuals). The IEEEbib.bst bibliography
% style file from IEEE produces unsorted bibliography list.
% -------------------------------------------------------------------------
\bibliographystyle{res/bib/IEEEbib}
\bibliography{res/bib/strings,res/bib/refs,res/bib/fields}

\end{document}
